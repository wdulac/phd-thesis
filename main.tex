\documentclass[12pt,twoside]{report}

\usepackage[no-math]{fontspec}
\setmainfont{Times New Roman}

%\usepackage[hidelinks]{hyperref}
\usepackage[french]{babel}
\usepackage[style=english]{csquotes}

\usepackage[authoryear]{natbib}
\bibliographystyle{abbrvnat}

\usepackage{graphicx} % Required for inserting images
\graphicspath{{Figures/}}

\usepackage{amsmath,amsfonts,amssymb,bm}
\usepackage{textcomp} %Required to use \textdegree
\usepackage{url}
\usepackage[locale=FR]{siunitx}

%\usepackage{tabularray}

%%%%%% Page style

\usepackage[a4paper,top=2.5cm, bottom=2cm, width=165mm, bindingoffset=6mm]{geometry}

\usepackage{fancyhdr}
\pagestyle{fancy}
\fancyhead{}
\fancyhead[LE]{\leftmark}
\fancyhead[RO]{\rightmark}
\setlength{\headheight}{29.8pt}
\addtolength{\topmargin}{-13.72pt}

\setlength{\parskip}{8pt}

%%%%%%%%%%%%%%%%%%%%%%%%%%%%%%%%%%%%%%%

\usepackage[french]{minitoc}
%\usepackage[nottoc]{tocbibind}
\usepackage{enumitem}
\setlist[itemize]{nosep, label={--}}
\usepackage[capitalize, nameinlink, noabbrev, french]{cleveref}

%%%%%% Biblatex settings

%\usepackage[backend=bibtex8, style=authoryear, autocite=inline, maxbibnames=20, minbibnames=20, backref=true, backrefstyle=three, date=year, isbn=false, giveninits=true, uniquelist=false]{biblatex} %uniquename=full
%\addbibresource{references.bib}

% \DefineBibliographyExtras{french}{\restorecommand\mkbibnamefamily} %pour que les noms des références ne soient pas en majuscules
% \renewbibmacro{in:}{ %eliminates the 'in' in the biblio
%     \ifentrytype{article}{}{%
%     \printtext{\bibstring{in}\intitlepunct}}}

%\setlength\bibitemsep{1.5\itemsep}
%\DeclareFieldFormat[inbook]{citetitle}{#1}
%\DeclareFieldFormat[inbook]{title}{#1}
%\DeclareFieldFormat[article]{citetitle}{#1}
%\DeclareFieldFormat[article]{title}{#1}

%%%%%%%%%%%%%%%%%%%%%%%%%%%%%%%%%%%

\usepackage{subfiles}

\title{Évolution future de l'activité cyclonique tropicale}
\author{William Dulac}
\date{2023}

%%%%%%%%%%%%%%%%%%%%%%%%%%%%%%%%%%%%
\begin{document}
\dominitoc

\chapter*{Remerciements}
Merci.

\chapter*{Résumé}
Ici sera le résumé

\chapter*{Abstract}
Abstract goes here

\tableofcontents

\chapter*{Introduction}
\addcontentsline{toc}{chapter}{Introduction}

%\mtcaddchapter{Introduction} % A quoi sert cette commande ?

\subfile{Chapitres/introduction}

\chapter{Cyclones tropicaux}
\subfile{Chapitres/chapitre1}

\chapter{Détection objective de cyclones tropicaux dans les simulations}
\subfile{Chapitres/chapitre2}

\chapter{Les indices de cyclogénèses}
\subfile{Chapitres/chapitre3}

\chapter{Application de l'indice en climat plus chaud}
\subfile{Chapitres/chapitre4}

\chapter*{Conclusion}
\addcontentsline{toc}{chapter}{Conclusion}
\subfile{Chapitres/conclusion}

% \appendix
% \chapter{Appendix Title}
% \subfile{Chapitres/appendix}

%\printbibliography
\bibliography{references}

\end{document}
