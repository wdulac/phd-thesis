% vim: spelllang=fr

\documentclass[12pt,twoside,openright]{book}

\usepackage{silence}
\WarningFilter{minitoc(hints)}{W0023}
\WarningFilter{minitoc(hints)}{W0028}
\WarningFilter{minitoc(hints)}{W0030}
\WarningFilter{minitoc(hints)}{W0024}

%%%%%%%%%% Figures and tables

%\usepackage{tabularray}

\usepackage{multirow}

\usepackage{graphicx} % Required for inserting images
\usepackage{adjustbox}
\graphicspath{
    {Figures/Chap1/}
    {Figures/Chap2/}
    {Figures/Chap3/}
    {Figures/Chap4/}
}

\usepackage{caption}
\captionsetup{
    labelfont=bf,
    labelsep=endash,
    margin=30pt
}

\usepackage{float}

\usepackage{pdfpages} % Pour importer mon article

\usepackage{makecell}
\usepackage{booktabs}

%%%%%% Biblatex settings

\usepackage[backend=biber, style=authoryear-comp, autocite=inline, maxbibnames=20, minbibnames=20, maxcitenames=2, backref=true, backrefstyle=three, date=year, isbn=false, giveninits=true, uniquelist=false, uniquename=init, natbib=false, sortcites=false]{biblatex} %uniquename=full

\DefineBibliographyExtras{french}{\restorecommand\mkbibnamefamily} %pour que les noms des références ne soient pas en majuscules
\renewbibmacro{in:}{ %eliminates the 'in' in the biblio
    \ifentrytype{article}{}{%
    \printtext{\bibstring{in}\intitlepunct}}}

% Apparemment, permet de ne pas avoir les titres d'article entre guillemets ?
\setlength\bibitemsep{1.5\itemsep}
\DeclareFieldFormat[inbook]{citetitle}{#1}
\DeclareFieldFormat[inbook]{title}{#1}
\DeclareFieldFormat[article]{citetitle}{#1}
\DeclareFieldFormat[article]{title}{#1}

% Supprime la date de consultation dans la bibliographie pour le(s) types indiqués.
\AtEveryBibitem{%
    \ifentrytype{article}{
            \clearfield{urlyear}
        }
    }

%%%%%%% Font and text related settings

\usepackage[no-math]{fontspec}
\usepackage[french]{babel}
\setmainfont[Ligatures=TeX]{Tex Gyre Termes}

\newfontfamily\greekfont[Scale=MatchUppercase,Ligatures=TeX]{Times New Roman}

\newcommand{\textgreek}[1]
    {\bgroup\greekfont{#1}\egroup} % Greek text

%\setmainfont[Ligatures=TeX]{Times New Roman}
\usepackage[style=french]{csquotes}

\def\frenchtablename{Tableau}

\usepackage{amsmath,amsfonts,amssymb,bm}
\usepackage{textcomp} %Required to use \textdegree
\usepackage{url}
\usepackage[locale=FR]{siunitx}
\sisetup{%
    mode=math,
    mathrm,
    detect-all,
}
\addto\extrasfrenchb{\sisetup{locale = FR}}
\DeclareSIUnit{\knot}{kt} % Remplace kn par kt, qui est plus communément utilisé (même si pas SI).
\usepackage[version=4]{mhchem} % Utilisé pour écrire 2 x CO2

%% Declare common units macros
\newcommand{\km}[1]{\SI{#1}{\kilo\metre}}
\newcommand{\Pa}[1]{\SI{#1}{\pascal}}
\newcommand{\hPa}[1]{\SI{#1}{\hecto\pascal}}
\newcommand{\m}[1]{\SI{#1}{\metre}}
\newcommand{\ms}[1]{\SI{#1}{\metre\per\second}}
\newcommand{\kmh}[1]{\SI{#1}{\kilo\metre\per\hour}}
\newcommand{\prct}[1]{\SI{#1}{\percent}}
\renewcommand{\ang}[1]{\SI{#1}{\textdegree}} % Surcharge la commande \ang existante pour avoir un espace entre la valeur et le symbole °
\newcommand{\mypm}[2]{$\num{#1}\pm\num{#2}$}
\usepackage{fnpct}

\usepackage[hidelinks]{hyperref}
\usepackage{xcolor}
\hypersetup{
    colorlinks, % Colours links instead of boxes
    %linkcolor={red!50!black},
    linkcolor=black, % Internal links
    %citecolor={blue!50!black}, % References
    citecolor=black,
    urlcolor={blue!80!black} % External links
}
\urlstyle{same}

\usepackage{lipsum}

%%%%%% Page style

\usepackage[a4paper, top=2.5cm, bottom=2cm, margin=2.5cm, bindingoffset=6mm]{geometry} % margin=2cm

\usepackage{fancyhdr}
\pagestyle{fancy}
\fancyhead{}
\fancyhead[LE]{\leftmark}
\fancyhead[RO]{\rightmark}
\setlength{\headheight}{29.8pt}
\addtolength{\topmargin}{-13.72pt}
%\renewcommand{\headrulewidth}{0pt} % Supprime la barre horizontale en haut de page de fancyhdr

\usepackage{emptypage} % Pour ne pas avoir de numéro et de header sur les pages vides

\setlength{\parskip}{8pt}
\interfootnotelinepenalty=10000 % Prevent page breaking in footnotes

%%%%%%%%%%%%%%%%%%%%%%%%%%%%%%%%%%%%%%%

\usepackage[nospace,french]{varioref}
\usepackage[nameinlink, noabbrev, french]{cleveref}
\usepackage[french]{minitoc}
\setcounter{secnumdepth}{3}
\renewcommand{\thesubsubsection}{\thesubsection.\Roman{subsubsection}}
\setcounter{minitocdepth}{3}

%\usepackage[nottoc]{tocbibind}
\usepackage{enumitem}
\setlist[itemize]{nosep, label={--}}

\usepackage{nextpage}

\usepackage{subfiles}
\addbibresource{\subfix{./references.bib}}

%%%%%%%%%%%%%%%%%%%%%%%%%%%%%%%%%%
%     Création 4ème couverture    %
%%%%%%%%%%%%%%%%%%%%%%%%%%%%%%%%%%

%{\newpage\vspace*{\fill}\thispagestyle{empty}\renewcommand{\headrulewidth}{0pt}}
%{\vspace*{\fill}}

%\newenvironment{vcenterpage}
%{\vspace*{\fill}\renewcommand{\headrulewidth}{0pt}}
%{\vspace*{\fill}}

%%%%%%%%%%%%%%%%%%%%%%%%%%%%%%%%%%%%
\begin{document}

\frontmatter

\includepdf{\subfix{./include/garde_tmp.pdf}}

\pagestyle{fancy}
\renewcommand{\sectionmark}[1]{%
\markboth{\thesection.\ \MakeUppercase{#1}}{}}

\dominitoc

%\chapter*{Remerciements}
%
%Merci.

\chapter*{Résumé}
\vspace{-1cm}

En raison de leurs impacts dévastateurs sur les populations et les infrastructures des pays concernés, l’évolution future de l’activité cyclonique tropicale
dans le contexte du réchauffement climatique est une question de grande importance. Deux méthodes existent pour évaluer l’activité cyclonique tropicale en
changement climatique dans les modèles de climat : l’utilisation d’algorithmes de détection (traqueurs) de cyclones ou l’utilisation d’indices de cyclogénèse 
qui traduisent des relations statistiques liant l’activité cyclonique observée à des variables atmosphériques de grande échelle. Ces deux méthodes tendent à
fournir des projections opposées dans les simulations climatiques. Motivée par ce désaccord, cette thèse propose alors d’explorer ces deux approches dans le but
d’apporter des améliorations à chacune d’elles.

Dans un premier temps, le traqueur de cyclones tropicaux du CNRM est appliqué à la réanalyse ERA5 du Centre européen pour les prévisions météorologiques à moyen
terme, et évalué à l’aide de la base de données d’observations des cyclones IBTrACS. Ses performances sont évaluées en termes de probabilité de détection et de
taux de fausses alarmes (POD et FAR), après optimisation des paramètres de détection, et en appliquant un filtre des systèmes de moyennes latitudes adéquat.
Plusieurs métriques d’évaluation de la similarité des trajectoires détectées dans ERA5 avec celles observées sont ensuite proposées puis comparées. Ces
métriques novatrices sont complémentaires au POD et au FAR et montrent que l’optimisation des paramètres de détection s’accompagne d’une légère amélioration de
la similarité des trajectoires.

De nouveaux indices de cyclogénèse sont ensuite construits sur ERA5 par régression de Poisson entre des prédicteurs de grande échelle thermiques et dynamiques,
et la base de données IBTrACS. Les régressions sont déclinées à différentes résolutions spatiales et temporelles ainsi qu’à l’échelle globale et pour les
différents bassins océaniques. La résolution temporelle accrue permet la correction du biais équatorial présent dans les indices les plus communément utilisés.
La variabilité interannuelle des indices apparaît cependant robuste aux modifications apportées aux coefficients de pondération des variables de grande échelle.
Suite à ce constat, l’apport de l’ajout de prédicteurs dans les régressions est évalué sur ERA5 ainsi que dans le modèle ARPEGE ; d’une part en ajoutant
explicitement un diagnostique du mode de variabilité El Niño (ENSO) dans l’indice, et d’autre part en remplaçant l’humidité relative à \hPa{600} par le déficit
de saturation d’humidité intégré sur la colonne (VPD). L’ajout du diagnostique ENSO permet alors d’améliorer la variabilité interannuelle de l’indice dans la
plupart des bassins océaniques. Les corrélations avec les séries observées est rendue statistiquement significative au seuil de \prct{95} dans tous les bassins
à l’exception du Nord Atlantique. L’utilisation du VPD permet quant à elle d’annuler les tendances à la hausse dans la période historique observée dans les
indices basés sur l’humidité relative. L’indice obtenu présente donc un meilleur accord avec les observations. Lorsqu’appliqué à des simulations climatiques
ARPEGE à très haute résolution, sous le scénario RCP8.5, le VPD amplifie également la diminution de l’activité cyclonique.

\chapter*{Abstract}

Given their devastating impact on the populations and infrastructures of the countries concerned the future evolution of tropical cyclone activity in the
context of global warming is an issue of great importance. Two methods exist for assessing tropical cyclone activity under climate change in climate models: the
use of cyclone detection algorithms (trackers) or the use of cyclogenesis indices, which translate statistical relationships linking observed cyclone activity
to large-scale atmospheric variables. These two methods tend to provide opposite projections in climate simulations. Motivated by this disagreement, this thesis
proposes to explore these two approaches, with the aim of making improvements to each.

Firstly, the CNRM tropical cyclone tracker is applied to the ERA5 reanalysis of the European Centre for Medium-Range Weather Forecasts, and evaluated using the
IBTrACS database of cyclone observations. Its performance is evaluated in terms of detection probability and false alarm rate (POD and FAR), after optimizing
detection parameters and applying an appropriate mid-latitude system filter. Several metrics for assessing the similarity of the tracks detected in ERA5 with
those observed are then proposed and compared. These innovative metrics are complementary to POD and FAR, and show that optimizing detection parameters is
accompanied by a slight improvement in track similarity.

New cyclogenesis indices are then constructed on ERA5 by Poisson regression between large-scale thermal and dynamic predictors, and the IBTrACS database. The
regressions are run at different spatial and temporal resolutions, as well as on a global scale and for different ocean basins. The increased temporal
resolution enables the equatorial bias present in the most commonly used indices to be corrected. However, the interannual variability of the indices appears to
be robust to changes in the weighting coefficients of the large-scale variables. Following this observation, the contribution of adding predictors to the
regressions is evaluated on ERA5 as well as in the ARPEGE model; on the one hand by explicitly adding a diagnostic of the El Niño (ENSO) variability mode to the
index, and on the other hand by replacing the relative humidity at \hPa{600} by the integrated moisture saturation deficit on the column (VPD). The addition of
ENSO diagnostics improves the interannual variability of the index in most ocean basins. Correlations with observed series are made statistically significant at
the \prct{95} threshold in all basins except the North Atlantic. The use of the VPD cancels out the upward trends in the historical period observed in indices
based on relative humidity. The resulting index is therefore in better agreement with observations. When applied to very high-resolution ARPEGE climate
simulations, under the RCP8.5 scenario, the VPD also amplifies the decrease in cyclonic activity.

\tableofcontents
\mainmatter

\chapter*{Introduction générale}
\markboth{INTRODUCTION GÉNÉRALE}{}
\addcontentsline{toc}{chapter}{Introduction générale}

\mtcaddchapter

\subfile{Chapitres/introduction}

\chapter{Cyclones tropicaux: État de l'art}\label{chap:chapitre_1}
\subfile{Chapitres/chapitre1}

\chapter{Algorithme de détection objective et de suivi de cyclones tropicaux}\label{chap:chapitre_2}
\subfile{Chapitres/chapitre2}

\chapter{Construction d'indices de cyclogénèse par régression statistique}\label{chap:chapitre_3}
\subfile{Chapitres/chapitre3}

\chapter{Introduction de nouveaux prédicteurs dans les indices}\label{chap:chapitre_4}
\subfile{Chapitres/chapitre4}

\chapter*{Conclusion et perspectives}
\markboth{CONCLUSION ET PERSPECTIVES}{} 
\addcontentsline{toc}{chapter}{Conclusion et perspectives}
\subfile{Chapitres/conclusion}

% \appendix
% \chapter{Appendix Title}
% \subfile{Chapitres/appendix}

\addcontentsline{toc}{chapter}{Bibliographie}
\printbibliography
%\bibliography{references}

\subfile{Chapitres/quatrieme_couverture.tex}

\end{document}
