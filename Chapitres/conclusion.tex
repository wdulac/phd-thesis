% vim: spelllang=fr

\documentclass[../main.tex]{subfiles}
\begin{document}

\section*{Contexte de la thèse}

Les cyclones tropicaux sont les phénomènes météorologiques naturels les plus extrêmes. Ils sont responsables de centaines de milliers de morts et les coûts des
dégâts matériels associés à leur passage se chiffrent en centaines de milliards de dollars. Le déséquilibre du système climatique causé par les émissions
anthropiques de gaz à effet de serre impacte les facteurs environnementaux favorables ou défavorables à la cyclogénèse, si bien que l'évolution future de
l'activité cyclonique tropicale est un sujet de grande importance. Le consensus scientifique du 6\ieme\ rapport d'évaluation du GIEC sur la question est que la
fréquence d'occurrence des cyclones tropicaux va probablement diminuer ou rester stable à l'échelle globale, tandis que la fréquence des cyclones les plus
intenses pourrait quant à elle augmenter. L'intensité maximale des qu'atteignent ces systèmes pourrait aussi augmenter, de même que les précipitations associées
aux cyclones tropicaux. À l'échelle des bassins océaniques individuels, l'incertitude associée à ces projections est cependant conséquente.

Deux méthodes existent pour évaluer l'activité cyclonique tropicale dans les simulations climatiques. La première consiste à utiliser un algorithme ou schéma de
détection de cyclones dans des simulations à haute résolution, tandis que la seconde consiste à utiliser des relations statistiques entre l'environnement de
grande échelle et l'activité cyclonique observée, pour ensuite les appliquer aux simulations climatiques. Ces deux approches convergent sur certains points,
notamment sur la probable augmentation de l'intensité maximale que les cyclones peuvent atteindre (via les projections futures dans l'intensité potentielle, ou
PI), mais tendent à indiquer des résultats divergeant quant au signe du changement dans la fréquence d'occurrence des cyclones, toutes catégories confondues.

Motivée par ce désaccord, cette thèse a porté sur l'étude de ces deux méthodes d'évaluation de l'activité cyclonique tropicale. Pour l'approche directe, les
objectifs étaient notamment de développer des métriques d'évaluation de la performance d'un schéma de détection, pour ensuite proposer de possibles
améliorations au traqueur du CNRM. Pour l'approche indirecte, l'accent fut principalement porté sur l'amélioration de la représentation de la variabilité
interannuelle inférée des indices de cyclogénèse.

\section*{Rappels des principaux résultats}

\subsection*{Étude de l'approche directe}

Pour étudier l'approche directe des schémas de détection et de suivi de cyclones tropicaux, le traqueur du CNRM est appliqué à la réanalyse ERA5, et les
trajectoires qui en sont issues sont comparées aux observations IBTrACS. Ce cadre de travail permet d'évaluer les performances du schéma de détection, notamment
en termes de probabilité de détection et de taux de fausses alarmes (POD et FAR), puisque les trajectoires historiques sont à priori présentes dans la
réanalyse. Ces deux métriques d'évaluation ont été utilisées pour caractériser la sensibilité du traqueur à ses divers seuils de détection. Bien que cette étude
ait été faite sur la réanalyse ERA5, elle donne néanmoins des indications plus générales sur la manière dont le schéma de détection est affecté par ces seuils.
Les deux paramètres les plus influents sur les performances du traqueur apparaissent alors être le seuil de vent maximal à la surface pour la probabilité de
détection, et le seuil d'anomalie de température par rapport à l'environnement pour le taux de faux positifs.

En plus de permettre l'évaluation du schéma de détection, ce cadre permet également d'évaluer la représentation des cyclones tropicaux dans la réanalyse ERA5.
Il est montré que la majorité des cyclones tropicaux historiques sont retrouvés dans la réanalyse. Toutefois, l'intensité maximale des vents cycloniques est
fortement sous-estimée, malgré la résolution très fine de \km{25} de la réanalyse. Les minimums de pression au niveau de la mer sont quant à eux mieux
représentés. En outre, la classe d'intensité des cyclones tropicaux dans la réanalyse ne correspond pas nécessairement à celle de leur équivalent dans les
observations, même en considérant le biais d'intensité du modèle. Il peut en effet arriver que certains cyclones soient plus intenses dans la réanalyse que dans
les observations.

Si, de manière générale, le réglage d'un paramètre pour augmenter l'efficacité de détection se traduit également par une augmentation du taux de fausses
alarmes, il est néanmoins possible de limiter efficacement le nombre de faux positifs en appliquant un filtre des systèmes de moyennes latitudes adéquat. Pour
pouvoir être utilisé dans des simulations climatiques futures, un tel filtre se doit d'être adaptatif, si bien qu'un simple seuil fixe en latitude n'est pas
recommandé. Plusieurs de ces filtres sont étudiés dans cette thèse, sous la forme de post-traitement des trajectoires issues du traqueur. L'un de ces filtres,
basé sur le paramètre $V_U^T$ de l'espace des phases de \textcite{hart_cyclone_2003} présente en outre la qualité de ne dépendre que des champs instantanés du
modèle, ce qui offre alors la perspective éventuelle d'une implémentation directement dans le schéma de détection.

La probabilité de détection et le taux de positifs sont des métriques n'apportant qu'une information limitée sur la capacité du traqueur à fidèlement détecter
les cyclones tropicaux. Un cyclone tropical observé peut en effet potentiellement être détecté au bon endroit et au bon moment dans la réanalyse et ainsi
compter favorablement dans le POD, tout en présentant dans la réanalyse une trajectoire d'apparence très différente de celle observée. Des métriques
complémentaires visant à évaluer quantitativement la similarité entre deux trajectoires sont alors développées. Ces métriques sont basées sur la notion de
similarité angulaire, définie en fonction de l'angle formé par les vecteurs de déplacements relatifs entre pas de temps consécutifs des deux trajectoires. Les
métriques proposées se distinguent par leur manière de comptabiliser les échéances pour lesquelles une seule des deux trajectoires est définie, c'est à dire
lorsque les deux trajectoires ne partagent pas exactement la même temporalité. 

Ces métriques sont mises en situation à travers une application visant à déterminer l'effet de l'optimisation des divers seuils de détection du traqueur du CNRM
---~faite dans le but de maximiser le POD sur ERA5~--- sur la similarité des trajectoires détectées avec les trajectoires historiques. L'analyse met alors en
évidence que cette optimisation s'est en fait accompagnée d'une légère amélioration de la similarité des trajectoires.

\subsection*{Étude de l'approche indirecte}

Pour étudier l'approche indirecte des indices de cyclogénèse, cette thèse s'est concentrée sur un indice en particulier les trois présentés, appelé TCS. Cette
indice a en effet la particularité d'être intégralement défini comme une régression de Poisson établie entre des variables thermiques et dynamiques de grande
échelle, et ajustée sur la densité de cyclogénèses, issue soit d'observations historiques ou bien d'un schéma de détection. Les autres indices couramment
utilisés possèdent, en comparaison, une part nettement plus importante de subjectivité et ne peuvent donc pas être aisément reproduits.

Dans un premier temps, la méthodologie de construction du TCS a alors été appliquée ici pour déterminer comment certaines modifications dans la manière de
réaliser la régression pouvait affecter les coefficients de pondération associés aux prédicteurs, et pour étudier comment ceux-ci impactaient en retour l'indice
de cyclogénèse en résultant. La réanalyse ERA5 a été utilisée pour déterminer l'effet de la résolution spatiale ainsi que de l'ajout de variabilité temporelle à
plus haute fréquence sur des indices ajustés sur la base de données IBTrACS. Il est alors montré que l'effet de l'ajout de variabilité temporelle ---~en
appliquant la régression sur les champs mensuels plutôt que sur les moyennes climatologiques~--- est dominant comparé à l'apport de la résolution spatiale
accrue. En outre, la modification du coefficient associé à la vorticité absolue à \hPa{850} par l'ajout de variabilité temporelle permet la correction quasi
intégrale du biais équatorial présent dans les trois indices présentes.

Il s'avère néanmoins que la modification des coefficients de la régression n'a que peu d'effet sur la représentation de la variabilité interannuelle simulée, ce
qui indique par conséquent que celle-ci est pilotée par le choix des variables thermiques et dynamiques utilisées. Un nouveau diagnostic est proposé, basé
sur le comportement de l'indice lors des cyclogénèses observées et consistant à étudier les paramètres de la distribution statistique des rangs de l'indice. Dit
autrement, ce diagnostic indique dans quelle mesure un indice de cyclogénèse s'active en lieu et date de la formation d'un cyclone dans les observations, par
rapport au reste du temps. Ce diagnostic montre alors qu'en dépit de l'absence d'amélioration dans la variabilité interannuelle, le fait de re-déterminer les
coefficients de la régression en vue d'appliquer l'indice aux données de son choix apporte tout de même une plus-value. L'intérêt de construire des indices de
cyclogénèse régionaux, c'est à dire dont les coefficients varient selon le bassin océanique, est également évalué. Les indices régionaux offrent les mêmes
performances à l'échelle interannuelle que leurs homologues globaux, mais offrent toutefois une cohérence sensiblement améliorée avec les densités observées, ce
qui est particulièrement intéressant dans les bassins de l'hémisphère nord, où la distribution méridienne de l'activité n'est pas la même dans toutes les
régions.

Dans un second temps, l'ajout de nouveaux prédicteurs dans la régression a été étudié. Un indice de cyclogénèse régional a été construit sur une simulation
historique, faite avec le modèle ARPEGE et forcée par les SST observée, en ajoutant comme prédicteur un diagnostic du mode de variabilité El Niño, connu
notamment pour son rôle de modulation de l'activité cyclonique. Il est alors montré que l'ajout de ce prédicteur permet d'améliorer la variabilité interannuelle
dans la plupart des bassins. La corrélation entre la variabilité interannuelle inférée de l'indice et celle mesurée par le traqueur du CNRM est rendue
statistiquement significative dans quasiment tous les bassins océaniques, quand bien même le prédicteur est rejeté par la régression. Le remplacement de
l'humidité relative à \hPa{600} par le déficit de saturation d'humidité intégré sur la colonne (VPD) est également évalué, sur la base de la littérature
indiquant une convergence du signal en changement climatique lorsque ce prédicteur est utilisé. Il est montré que l'utilisation du VPD en lieu et place de
l'humidité relative réduit très fortement le biais positif présent en Mer Rouge et dans le Golfe Persique présent dans tous les indices utilisant l'humidité
relative.

Des tendances sont recherchées dans l'évolution du nombre d'occurrence annuel de cyclones tropicaux inféré de différents indices de cyclogénèse appliqués à la
simulation ARPEGE historique ainsi que dans la réanalyse ERA5. La problématique de la divergence entre les approches directes et indirectes dans les projections
climatiques est visible dans ces séries temporelles, en ce sens que les indices de cyclogénèses basés sur l'humidité relative indiquent une tendance à la hausse
statistiquement significative, à l'échelle globale et dans plusieurs bassins océaniques. De telles tendances ne sont pas détectées dans les observations ou
dans les trajectoires du modèle. Les indices utilisant le VPD ne présentent pas de tendances à la hausse non plus.

Enfin, l'effet du VPD est évalué sur deux simulations climatiques à climat futur sous le scénario RCP8.5 avec ARPEGE dans sa configuration basculée-étirée : une
pour le bassin Nord Atlantique et l'autre le bassin Sud Indien. La résolution horizontale de ces simulations atteint localement \km{14} de résolution
horizontale au pôle de la grille situé au milieu du bassin considéré. Si, dans ces simulations et dans ces bassins, tous les indices considérés projettent une
diminution de l'activité cyclonique du même ordre de grandeur que la détection objective des cyclones dans ces simulations, l'indice utilisant le VPD produit
une diminution plus prononcée et plus homogène dans l'espace. La réponse suggérée par le TCS dans ces simulations présente un bon accord avec la réponse
indiquée par le schéma de détection, aussi bien dans son amplitude que dans la répartition spatiale du changement, notamment pour la simulation sur le bassin
Sud Indien.

\subsection*{Perspectives pour des travaux futurs}

Le traqueur du CNRM offre des performances satisfaisantes, mais celui-ci bénéficierait sans aucun doute d'un travail de refonte. En effet, son utilisation est
complexe et nécessite l'ajustement systématique de certains paramètres directement dans les sources avant application. Par ailleurs, l'utilisation d'un filtre
des systèmes de moyennes latitudes réduit considérablement les détections indésirables, et le traqueur gagnerait donc à ce qu'un tel système y soit
directement implémenté. Le filtre du paramètre $V_U^T$ est un bon candidat pour cette implémentation, mais les deux paramètres \textbf{PT} et \textbf{PW}
deviendraient alors redondants.

De plus, l'implémentation qui est faite du filtre du $V_U^T$ dans \textcite{dulac_assessing_2023} est perfectible sur plusieurs aspects. Le premier point,
mineur, est que le gradient vertical du maximum de perturbation isobarique y est simplement calculé à partir de la différence entre les niveaux \hPa{600} et
\hPa{300}. Bien que des essais préliminaires montrent que les différences sont faibles, il serait néanmoins plus rigoureux d'évaluer le gradient vertical sur
l'ensemble des niveaux verticaux intermédiaires, via un ajustement linéaire du profil vertical de la perturbation de géopotentiel, tel que
\textcite{hart_cyclone_2003} le suggère. Le second point concerne le rayon de recherche autour du centre du TC pour la mesure de la perturbation maximale de
géopotentiel. Celui-ci est fixé à \km{500} autour du centre. Ce rayon est justifié dans \textcite{hart_cyclone_2003} comme étant le rayon moyen dans lequel la
circulation cyclonique secondaire (convergente) est constatée dans les observations \parencite{frank_structure_1977}. Le rayon de recherche affecte les valeurs
du paramètre $V_U^T$, et la distance idéale dépend en fait de la taille du système considéré. Une valeur non-appropriée peut aboutir à l'élimination d'un TC par
le filtre ---~ou au contraire à sa préservation~--- pour de mauvaises raisons. Le traqueur du CNRM contient déjà des routines efficaces de calcul de la taille
du cyclone, si bien que leur utilisation comme étape préalable au diagnostic de vent thermique, dans le cadre d'une implémentation online de ce filtre,
permettrait de s'assurer que le rayon de recherche est toujours approprié au système considéré.

Les métriques d'évaluation de la similarité des trajectoires présentées dans la \cref{sec:similarité} gagneraient également à être retravaillées, notamment car
toutes les métriques proposées sont basées sur la moyenne de la similarité angulaire, celle-ci étant calculée entre chaque paire de vecteurs de déplacements
relatifs. Par conséquent, les scores sont peu sensibles à des divergences locales, en particulier lorsque les trajectoires évaluées sont longues. Les
trajectoires cycloniques peuvent aisément contenir des dizaines de pas de temps, plus encore si le schéma de détection est appliqué à des données de
fréquence supérieure à \num{6} heures, et donc la moyenne n'est à priori pas l'opérateur le plus adapté pour mesurer la similarité de la trajectoire dans son
ensemble.

Les métriques de similarité pourraient peut-être fournir une information sur l'effet du changement climatique sur la forme des trajectoires. Une stratégie pour
évaluer cet effet pourrait consister à évaluer la similarité de couples de trajectoires formés aléatoirement dans des simulations à climat présent, puis à
répéter l'opération sur des trajectoires issues de simulations à climat futur. Il serait alors peut-être possible de déterminer si le réchauffement climatique
amène à une plus grande variété de formes de trajectoires différentes, ou au contraire à des trajectoires plus semblables les unes avec les autres. Une telle
stratégie d'évaluation pourrait néanmoins être sensible au décalage vers les pôles des TC, auquel cas il serait peut-être nécessaire de réaliser cette analyse
par bande de latitude. Outre la recherche en changement climatique, les métriques de similarité des trajectoires pourraient également avoir des applications
concrètes en mode prévision. Par exemple, ces dernières pourraient être utilisées pour contraindre le cône d'incertitude des trajectoires de TC pour les
activités opérationnelles de veille cyclonique.

Les travaux réalisés dans cette thèse sur les indices de cyclogénèse ont essentiellement porté sur la recherche de l'amélioration de la variabilité
interannuelle, mais seul l'apport artificiel de l'information concernant la phase de l'ENSO a permis d'obtenir une amélioration, certes significative, mais
modérée. Dans la mesure où les prédicteurs utilisés contrôlent la variabilité interannuelle de l'indice, les performances relativement mauvaises obtenues
peuvent s'expliquer de plusieurs manières, lesquelles ne sont d'ailleurs pas nécessairement mutuellement exclusives. Il est en effet possible que le processus
physique de la cyclogénèse ne soit pas adéquatement capturé par ces combinaisons de prédicteurs, autrement dit que les bons prédicteurs sont absents. Il est
également possible que la variabilité interannuelle ne réponde pas à un forçage, mais plutôt que le processus de cyclogénèse possède une dimension
intrinsèquement stochastique. Cette perspective questionne la définition même d'un indice de cyclogénèse. Si ces derniers ne font que renseigner le degré avec
lequel l'environnement est favorable à la formation de cyclones, mais que la cyclogénèse en elle-même est en fin de compte aléatoire, alors les indices n'ont
aucune raison de simuler une variabilité interannuelle proche des observations.

Une troisième possibilité existe néanmoins, car il se peut également que le modèle statistique utilisé ici, à savoir la régression de Poisson, ne soit pas
adéquat pour reproduire la variabilité interannuelle. Le fait que l'ajout du diagnostic ENSO ait présenté une amélioration de la variabilité interannuelle en
dépit du rejet du prédicteur par la régression tend en effet à suggérer que des modèles différents pourraient offrir de meilleures performances dans cette
optique. Il serait par conséquent intéressant de chercher à mettre en place des modèles statistiques plus sophistiqués. En particulier, il existe des modèles
qui fonctionnent en deux temps (de type Hurdle), dans lesquels une première composante pourrait avoir pour fonction ---~dans le cas qui nous intéresse~--- de
déterminer s'il y a, oui ou non, formation d'un cyclone tropical, tandis que la deuxième composante aurait pour fonction d'en déterminer le nombre. Cette
catégorie de modèles permet en outre de mieux prendre en considération l'excès zéros dans les densités de cyclogénèses observées, servant de prédictant à la
régression. En effet, la plupart des points de grille ne présentent le plus souvent aucun cyclone. 

Il convient de rappeler qu'historiquement, les indices de cyclogénèse étaient utilisés pour étudier l'activité cyclonique dans des modèles dont la résolution
spatiale ne permettaient pas de simuler de TC. Avec l'arrivée du premier ensemble multi-modèles à haute résolution (HighResMIP), dont certains modèles
atteignent jusqu'à \km{20} de résolution, il devient légitime de questionner la pertinence de l'utilisation d'indices par rapport à la mesure directe des TC
dans les simulations par détection objective. Mais si ces hautes résolutions permettent en effet de simuler des TC, et parfois même d'atteindre les catégories 4
et 5 sur l'échelle de Saffir-Simpson, ces modèles sont encore peu nombreux, et leur utilisation particulièrement coûteuse. Le protocole HighResMIP précaunise à
ce titre des simulations plus courtes que dans CMIP6 \parencite[et plus particulièrement ScenarioMIP,][]{oneill_scenario_2016}, et sous un seul scénario de
réchauffement global. HighResMIP ne constitue alors qu'un sous-ensemble de taille moindre et de portée réduite par rapport à l'ensemble mutli-modèles CMIP6. Les
indices sont donc encore pertinents pour beaucoup de modèles CMIP6, dont la résolution spatiale atteint jusqu'à \km{250}. Par ailleurs, l'application d'un
indice de cyclogénèse à une simulation est nettement plus aisée que le déploiement d'un schéma de détection et suivi objectif, lequel apporte nécessairement de
la complexité additionnelle, et nécessite de traiter les sorties des modèles à une fréquence sub-journalière.

La validité de l'utilisation d'indices de cyclogénèse comme méthode d'évaluation de l'activité cyclonique tropicale dans des simulations climatiques dépend de
l'hypothèse selon laquelle les relations statistiques entre l'activité observée et l'environnement de grande échelle sont stationnaires dans le temps. La
reproduction de la méthodologie de \textcite{tippett_poisson_2011} a mis en évidence une sensibilité des coefficients de la régression à la période de référence
utilisée pour les déterminer, et notamment une possible dérive des coefficients dans le temps. En particulier, les coefficients obtenus dans le cadre du
protocole C25 semble s'éloigner de ceux du TCS si la période de référence est distante de celle utilisée par \textcite{tippett_poisson_2011}. Ce constat ne
constitue cependant qu'un résultat extrêmement préliminaire. Il serait particulièrement intéressant de chercher à étudier ce phénomène plus en détails, par
exemple en analysant les variations des coefficients de la régression obtenus par glissement de la période de référence.

Enfin, les prédicteurs utilisés dans la formulation des indices construits dans cette thèse ont été sélectionnés sur la base de la littérature existante dans le
cas du VPD, et pour le rôle connu de l'ENSO à moduler l'activité cyclonique tropicale. Une sélection plus large et objective des prédicteurs pourrait être
envisagée, à la manière des travaux de \textcite{wang_dynamic_2020}.

\end{document}
