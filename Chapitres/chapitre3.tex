% vim: spelllang=fr

\documentclass[../main.tex]{subfiles}
\graphicspath{{\subfix{../Figures/Chap3/}}}
\begin{document}

\begin{itshape}
    Ce troisième chapitre porte sur les indices de cyclogénèses, de la formulation précise des différents indices utilisés classiquement dans la littérature à
    des indices définis pour les besoins présents, et de leur application à ERA5. [en cours]
\end{itshape}

\minitoc
\newpage
%--------------------------------------
\section{Les indices de cyclogénèses}

\subsection{Tour d'horizon des différents indices}

La \cref{sec:intro_indices} du \cref{chap:chapitre_1} introduit le concept de l'indice de cyclogénèse comme un moyen indirect d'étudier l'activité cyclonique
dans les modèles. Cette approche se place comme une alternative à la mesure directe de la détection et le suivi objectif de TC dans les simulations, introduite
dans la \cref{sec:intro_tracking} et abordé plus amplement dans le \cref{chapitre_2}. Ce thème de recherche a été initié par les travaux de
\textcite{gray_global_1968,gray_tropical_1975} qui a identifié les facteurs environnementaux de grande échelle les plus favorables à la cyclogénèse pour ensuite
les combiner en un produit capable de reproduire la densité climatologique de cyclones observés, avec son indice alors dénommé Paramètre de Genèse Saisonnière,
ou SGP. Ce dernier (de même que sa variante annuelle le YGP) n'est toutefois plus d'usage aujourd'hui, à fortiori pour des applications en changement
climatique, à cause de son seuil fixe de SST. Une certaine variété d'indices ont été développés depuis. Ces derniers peuvent être distingués selon si les
coefficients de pondération associés à chaque prédicteur sont déterminés de manière partiellement empirique, de ceux issus de régression statistiques et qui
sont donc plus facilement reproductibles. Les plus communément utilisés d'entre eux sont présentés ci-après.

\subsubsection{Relations empiriques}

Déjà mentionné dans la \cref{sec:intro_indices}, le premier indice de cyclogénèse qui a suivi le SGP de \citeauthor{gray_tropical_1975} est le CYGP
(\textit{Convective Yearly Genesis Parameter}) \parencite{royer_gcm_1998}. Ce dernier a été conçu pour résoudre le problème de la dépendance du SGP à un seuil
fixe de SST et se définit comme suit.
%
\begin{equation*} \mathrm{CYGP} = \underbrace{\lvert f \rvert \, I_\zeta \, I_S}_{\mathrm{Dynamique}} \, \underbrace{\vphantom{I_\zeta}k(P_c -
P_0)}_{\mathrm{Convectif}} \tag*{CYGP}
\end{equation*}
%
Où $f = 2 \Omega \sin \phi$ est le paramètre de Coriolis, $I_\zeta = \zeta_R \! +\! 5$ où $\zeta_R$ est la vorticité relative, $I_S = (1/S_z \! + \! 3)$, où
$S_z$ est le cisaillement vertical du vent entre le haut et le bas de la troposphère, $P_c$ sont les précipitations convectives sur océan seuillées sur $P_0$,
et enfin $k$ étant le coefficient de calibration de l'indice. Les constantes ajoutées au cisaillement et à la vorticité relative sont celles du SGP originel,
mentionnées dans la \cref{sec:intro_indices}, \cref{note:SGP} et constituent là le caractère empirique de l'indice. Comme son nom le suggère, la particularité
du CYGP est d'utiliser un potentiel convectif en lieu et place du potentiel thermique du SGP qui, dans \textcite{gray_tropical_1975} est défini avec la
stabilité atmosphérique (gradient vertical de température virtuelle équivalente), l'humidité relative moyenne entre \hPa{700} et \hPa{500} et l'énergie
thermique océanique accumulée sur les 60 premiers mètres sous la surface. Le CYGP de \citeauthor{royer_gcm_1998} a servi dans des applications en climat futur
par \textcite{mcdonald_tropical_2005,cattiaux_projected_2020,chauvin_response_2006,chauvin_future_2020}

L'indice le plus communément utilisé est sans aucun doute le GPI (\textit{Genesis Potential Index}) de \textcite{emanuel_tropical_2004} défini comme suit :
%
\begin{equation*}
    \tag*{GPI}
    \mathrm{GPI} = \underbrace{\vphantom{\left( \frac{H}{50} \right)^3}\lvert 10^5 \zeta \rvert^{3/2} \, (1 + 0.1 V_{\mathrm{shear}})^{-2}}_{\mathrm{Dynamique}}
    \underbrace{\left( \frac{H}{50} \right)^3 \left( \frac{V_{\mathrm{pot}}}{70} \right)^3}_{\mathrm{Thermique}}
\end{equation*}
%
Où $\zeta$ représente la vorticité absolue à \hPa{850} ($\zeta = \zeta_R + f$) exprimée en 10$^{-5}$~s$^{-1}$, $V_{\mathrm{shear}}$ l'amplitude du cisaillement
vertical du vent entre \hPa{850} et \hPa{200}, $H$ l'humidité relative à \hPa{700} et $V_{\mathrm{pot}}$ l'intensité potentielle (\textit{Potential Intensity},
PI) \parencite{emanuel_airsea_1986,emanuel_sensitivity_1995,bister_dissipative_1998,bister_low_2002}. Cette dernière quantité remplace la SST dans l'indice et
représente l'intensité maximale théorique d'un cyclone tropical, exprimée en vitesse de vent à la surface (\ms{}), lorsque ce dernier est assimilé à un cycle de
Carnot \parencite{emanuel_dependence_1987}. Sous cette hypothèse, le PI est proportionel à l'efficacité énergétique du cycle thermodynamique ---~exprimée par la
différence de température à ses deux extrémités~--- et de la source de chaleur. \textcite{bister_low_2002} ont alors montré que, sous certaines hypothèse
supplémentaires dont l'énumération dépasse le cadre de ce manuscrit, le PI pouvait être évalué comme suit :
%
\begin{equation*}
    \tag*{PI}
    V_{\mathrm{pot}}^2 = \frac{T_s}{T_0} \frac{C_k}{C_D} \left[ \mathrm{CAPE}^* - \mathrm{CAPE} \right] \rvert_{\mathrm{RMW}}
\end{equation*}
%
Où $T_s$ est la température de surface, $T_0$ la température de l'air rejeté au sommet du cyclone, $C_k$ est le coefficient d'échange d'enthalpie avec l'océan,
$C_D$ le coefficient de trainée, CAPE$^*$ l'énergie potentielle convective disponible (\textit{Convective Available Potential Energy}) de l'air saturé et élevé
jusqu'au niveau de l'air expulsé au sommet du cyclone, et CAPE étant l'énergie potentielle convective de la couche limite atmosphérique ---~CAPE$^*$ et CAPE
sont tous deux évalués au rayon de vent maximum (\textit{Radius of Maximum Wind}, RMW). En pratique, le calcul de l'intensité potentielle nécessite
l'utilisation d'un algorithme sophistiqué dont l'exécution nécessite un temps de calcul conséquent.

Plusieurs autres indices de cyclogénèses ont été dérivés du GPI. \textcite{emanuel_tropical_2010} a lui-même proposé une révision du GPI originel, remplaçant
l'humidité relative par une mesure adimensionnelle du déficit de saturation d'entropie humide dans la moyenne troposphère, argumentant que cette quantité serait
un meilleur prédicteur de la fréquence d'occurrence des TC \parencite{emanuel_hurricanes_2008}. \textcite{bruyere_investigating_2012} ont également proposé une
version modifiée du GPI originel dans laquelle la vorticité et l'humidité relative sont supprimées. Les auteurs montrent en effet que les variations
interannuelles de ces deux quantités sont quasi inexsistantes et ne peuvent donc être liées à la fréquence d'occurrence des TC, et suggèrent plutôt que la
contribution de ces deux prédicteurs consistent en des seuils minimaux nécessaires \parencite{mcgauley_measuring_2011}. \citeauthor{bruyere_investigating_2012}
suggèrent également que le PI puisse jouer le rôle de proxy de la capacité de l'environnent à favoriser la convergence d'humidité. À l'échelle intra-saisonnière
cette fois, les résultats de \textcite{camargo_diagnosis_2009} arrivent à un constat en parfaite opposition puisque les auteurs montrent que l'humidité relative
à \hPa{850} et la vorticité absolue sont respectivement les deux facteurs prépondérants dans les anomalies du GPI liées à l'oscillation de Madden-Julian
(\textit{Madden-Julian Oscillation}, MJO), mode de variabilité principale à cette échelle de temps. \textcite{wang_dynamic_2020} proposent quant à eux un indice
composé uniquement de prédicteurs dynamiques, sans aucune contribution thermique, et montrèrent que la capacité de leur produit à simuler la variabilité
interannuelle dans le bassin WPac et dans l'hémisphère sud est améliorée par rapport au GPI. Cette grande variété d'indices est représentative de l'incertitude
qui entoure le choix des prédicteurs dans la formulation des indices de cyclogénèses, choix notamment compliqué par le fait que les paramètres de grande
échelles qui influent sur la fréquence d'occurrence dépendent de l'échelle temporelle considérée \parencite{wang_anomalous_2017}.

\subsubsection{Modélisation statistique de l'activité cyclonique}

Les indices introduits dans la section précédente sont généralement déterminés de manière semi empirique à l'aide de considérations théoriques et en s'appuyant
sur des régressions multiples entre différents champs modèles, provenant souvent de réanalyses, et l'activité observée. Les relations ainsi établies sont
ensuite appliquées plus largement sur toutes sortes de simulations. La partie subjective derrière la mise au point des indices empêche la reproductibilité des
résultats ayant conduits à leur formulation, ce qui contribue probablement à l'émergence de nouvelles variantes de ces mêmes indices, répétant ainsi le même
schéma. C'est entre autre pour cette raison que \textcite{tippett_poisson_2011} proposèrent une approche différente consistant à définir un indice de
cyclogénèse purement comme une régression non-linéaire entre l'activité observée et l'environnement, avec une méthodologie objective de sélection des
prédicteurs. L'indice de Tippett, par la suite référé par \textquote{Tipp} ou encore \textquote{TCS}, du nom des trois auteurs (Tippett, Camargo, Sobel) est
défini comme suit :
%
\begin{equation*}
    \tag*{TCS}
    \mathrm{TCS} = \exp \Big( b + \underbrace{b_\eta \eta + b_{V_{\mathrm{shear}}} V_{\mathrm{shear}}}_{\mathrm{Thermique}} + \underbrace{\vphantom{b_\eta}b_H H + b_T
    T}_{\mathrm{Dynamique}} + \log \cos \phi \Big)
\end{equation*}
%
Où $\eta = \min (\zeta, \num{3.7})$, vorticité absolue à \hPa{850} exprimée en 10$^{-5}$~s$^{-1}$ et bornée à \SI{3.7e-5}{\per\second} (\textit{clipped
vorticity}), $V_{\mathrm{shear}}$ le cisaillement vertical entre \hPa{850} et \hPa{200} exprimé en \ms{}, $H$ l'humidité relative à \hPa{600} en pourcentage et
$T$ la SST relative définie comme l'écart de température à la SST moyennée entre \ang{20}S et \ang{20}N sur océan. Dans \textcite{tippett_poisson_2011}, cet
indice est construit par une régression de Poisson entre la climatologie mensuelle des cyclogénèses IBTrACS, définies comme les premières observations des
systèmes atteignant au moins le stade de tempête tropicale, et la climatologie mensuelle des prédicteurs calculée entre \num{1961} et \num{2000} par le biais de
la réanalyse ERA-40 \parencite{uppala_era40_2005}. Les coefficients de la régression proposée par \citeauthor{tippett_poisson_2011} sont $b = \num{-5.80}$,
$b_{V_{\mathrm{shear}}} = \num{-0.15}$, $b_H = \num{0.05}$ et $b_T = \num{0.56}$. La formulation du prédicteur de vorticité utilisé dans le TCS est différente
des autres indices puisqu'elle utilise une borne supérieure au tourbillon absolu. \textcite{tippett_poisson_2011} justifient cette approche ainsi que le seuil
de \SI{3.7e-5}{\per\second} en montrant que le nombre de cyclogénèses n'est plus influencé par la vorticité au delà de cette valeur. L'utilisation d'un
prédicteur borné de vorticité absolue à \hPa{850} permet alors au TCS d'être plus sensible aux variations de vorticité dans les tropiques que si la vorticité
non bornée était considérée, tout en réduisant le nombre de cyclogénèses simulé dans les hautes latitudes. En outre, c'est par cet effet de seuil de la
vorticité que le retrait de ce prédicteur est justifié dans l'indice de \textcite{bruyere_investigating_2012}. \citeauthor{tippett_poisson_2011} indiquent
également que l'utilisation de la SST relative plutôt que du PI comme prédicteur de l'énergie thermique océanique disponible produit une régression d'une
qualité légèrèrement supérieure. L'intérêt principal réside cependant dans l'interprétabilité facilitée par la SST relative, le PI n'étant à priori pas relié à
la cyclogénèse. D'un point de vue pratique, si les deux prédicteurs peuvent être interchangés dans le TCS, en remplaçant également $b_T$ par une valeur
$b_{\mathrm{PI}}$ appropriée, comme par exemple dans \textcite{camargo_testing_2014,camargo_tropical_2016}, étant donné que les deux variables sont largement
corrélées \parencite{vecchi_effect_2007,swanson_nonlocality_2008}, la SST relative est cependant considérablement plus simple à calculer.
\textcite{tippett_poisson_2011} montrent enfin que leur indice produit une description de l'activité cyclonique améliorée sur certains aspects par rapport au
GPI. En particulier, le GPI a tendance à simuler une activité trop forte en dehors de la saison cyclonique active, et la différence entre les deux hémisphères
n'est pas assez prononcée. Le TCS simule également mieux l'étendue méridionale de l'activité cyclonique par rapport au GPI ---~notamment grâce à l'usage de la
vorticité bornée~--- ce dernier produit en effet une activité trop prononcée proche de l'équateur et dans les hautes latitudes.

\textcite{wang_dynamic_2020} ont également défini un indice de cyclogénèse par un modèle statistique, cette fois ci par une régression log-log\footnote{Une
régression de Poisson peut être vue comme une relation log-linéare, reliant linéairement le logarithme de l'espérance de la quantité $Y$ aux prédicteurs
$\mathbf{x}$ par le vecteur des coefficients $\mathbf{b}$ : $\log \mathrm{E} (Y \mid \mathbf{x}) = \mathbf{b}^{\mathrm{T}} \mathbf{x}$, tandis que la
régression utilisée par \textcite{wang_dynamic_2020} est de type $\log \mathrm{E}(Y \mid \mathbf{x}) = \mathbf{b}^{\mathrm{T}} \log \mathbf{x}$. Bien que
les auteurs y réfèrent également sous l'appellation de log-linéaire, nous l'appelons ici log-log pour la distinguer de la régression pratiquée par
\citeauthor{tippett_poisson_2011}.} ainsi qu'une méthodologie de sélection automatique des prédicteurs parmi \num{11} candidats, comparable à celle utilisée par
\textcite{tippett_poisson_2011}, sur cinq réanalyse, à l'échelle globale et pour chaque hémisphère. \citeauthor{wang_dynamic_2020} montrent alors que les
prédicteurs dynamiques sont systématiquement sélectionnés, amenant à la définition d'un indice purement dynamique, nommé DGPI (\textit{Dynamic}). En effet, dans
\textcite{tippett_poisson_2011}, il est considéré comme acquis que l'indice final se doit d'être une combinaison de prédicteurs thermiques et dynamiques,
s'inscrivant dans la continuité des travaux de \textcite{gray_tropical_1975}. Dans ce contexte, le processus de sélection consiste à utiliser l'estimation de la
vraisemblance log (\textit{log likelihood}), que la régression vise à maximiser, et ajustée par le nombre de prédicteurs pour compenser le biais positif
introduit par l'ajout de variables additionnelles \parencite{akaike_information_1998}, pour ensuite déterminer duquel des deux prédicteurs thermiques entre la
SST relative et le PI produit la meilleure régression. Ce processus est répété pour la vorticité absolue avec la vorticité absolue bornée, ainsi que pour
l'humidité relative à \hPa{600} et l'humidité relative intégrée sur la colonne atmosphérique. La place du cisaillement vertical dans la régression n'est pas
remise en cause \parencite{gray_global_1968}. \textcite{wang_dynamic_2020} utilisent quant à eux une régression séquentielle évaluant l'apport de l'ajout de
chaque variable parmi les \num{11}, attribuant à chacun d'entre eux un score dont la valeur dépend de l'ordre dans lequel ils sont sélectionnés, et ce jusqu'à
ce que la significativité de l'apport des nouveaux prédicteurs, évaluée par f-test, ne puisse plus être vérifiée. C'est cette méthodologie qui a conduit à la
définition de leur indice constitué des prédicteurs dynamiques suivants, avec des scores du plus élevé au plus faible : le cisaillement vertical entre \hPa{850}
et \hPa{200} en \ms{}, la vitesse verticale du vent à \hPa{500} en \SI{}{\pascal\per\second}, la vorticité absolue à \hPa{850} en \SI{}{\per\second} et enfin le
gradient méridional de la composante zonale du vent à \hPa{500} en \SI{}{\per\second}
%
\begin{equation*}
    \tag*{DGPI}
    \begin{split}
        \mathrm{DGPI} = (\num{2,0} + & \num{0,1} \times V_{\mathrm{shear}})^{-1} \left( \num{5,5} - \frac{d u_{500}}{dy} \times 10^5 \right)^{2}\\
                                     &(\num{5,0} - \num{20} \times \omega_{500})^3 (\num{5,5} + \lvert \zeta_{850} \times 10^5 \rvert )^2 e^{-12} - \num{1.0}
    \end{split}
\end{equation*}
%
Dans le DGPI, les constantes ajoutées aux prédicteurs servent de normalisation de façon à ce que l'amplitude de leurs variations soient comparables, une fois
transformées en logarithmes,  Les auteurs précisent toutefois que la définition purement dynamique de l'indice ne signifie pas que les variables thermiques
n'importent pas dans les processus physique sous-jacents de la cyclogénèse, mais suggèrent plutôt que l'influence de ces derniers est suffisamment bien
représentée dans les variables dynamiques. Malgré les choix et les pondérations différentes entre l'indice de \textcite{wang_dynamic_2020} et le GPI, les
performances évaluées sur le climat présent (\num{1980}~--~\num{2017}) sont comparables entre les deux indices. En particulier, \textcite{wang_dynamic_2020}
soulignent que la corrélation de la variabilité interannuelle du DGPI avec les observations est améliorée dans le bassin WPac et dans l'hémisphère sud, mais
dégradée dans le bassin EPac.

%--------------------------------------
\section{Exploration du potentiel prédictif d'un indice à l'échelle saisonnière}

\subsection{Choix du pas d'agrégation temporelle : Mensuel versus saisonnier}

\subsection{Effet de la résolution spatiale et temporelle}\label{sec:resolution_spatiale_temporelle}

\subsection{Activité observée versus modèle ARPEGE}

\subsection{Scores de performances sur la variabilité inter-annuelle}


%--------------------------------------
\section{Synthèse}

\end{document}
