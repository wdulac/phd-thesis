% vim: spelllang=fr

\documentclass[../main.tex]{subfiles}
\graphicspath{{\subfix{../Figures/Chap3/}}}
\begin{document}

\begin{itshape}
    Ce troisième chapitre porte sur les indices de cyclogénèses, de la formulation précise des différents indices utilisés classiquement dans la littérature à
    des indices définis pour les besoins présents, et de leur application à ERA5. [en cours]
\end{itshape}

\minitoc
\newpage
%--------------------------------------
\section{Les indices de cyclogénèses}

\subsection{Tour d'horizon des différents indices}

La \cref{sec:intro_indices} du \cref{chap:chapitre_1} introduit le concept de l'indice de cyclogénèse comme un moyen indirect d'étudier l'activité cyclonique
dans les modèles. Cette approche se place comme une alternative à la mesure directe de la détection et le suivi objectif de TC dans les simulations, introduite
dans la \cref{sec:intro_tracking} et abordé plus amplement dans le \cref{chapitre_2}. Ce thème de recherche a été initié par les travaux de
\textcite{gray_global_1968,gray_tropical_1975} qui a identifié les facteurs environnementaux de grande échelle les plus favorables à la cyclogénèse pour ensuite
les combiner en un produit capable de reproduire la densité climatologique de cyclones observés, avec son indice alors dénommé Paramètre de Genèse Saisonnière,
ou SGP. Ce dernier (de même que sa variante annuelle le YGP) n'est toutefois plus d'usage aujourd'hui, à fortiori pour des applications en changement
climatique, à cause de son seuil fixe de SST. Une certaine variété d'indices ont été développés depuis. Ces derniers peuvent être distingués selon si les
coefficients de pondération associés à chaque prédicteur sont déterminés de manière partiellement empirique, de ceux issus de régression statistiques et qui
sont donc plus facilement reproductibles. Les plus communément utilisés d'entre eux sont présentés ci-après.

\subsubsection{Relations empiriques}

Déjà mentionné dans la \cref{sec:intro_indices}, le premier indice de cyclogénèse qui a suivi le SGP de \citeauthor{gray_tropical_1975} est le CYGP
\parencite{royer_gcm_1998}. Ce dernier a été conçu pour résoudre le problème de la dépendance du SGP à un seuil fixe de SST et se définit comme suit.
%
\begin{equation*} \mathrm{CYGP} = \underbrace{\lvert f \rvert \, I_\zeta \, I_S}_{\mathrm{Dynamique}} \, \underbrace{\vphantom{I_\zeta}k(P_c -
P_0)}_{\mathrm{Convectif}} \tag*{CYGP}
\end{equation*}
%
Où $f = 2 \Omega \sin \phi$ est le paramètre de Coriolis, $I_\zeta = \zeta_R \! +\! 5$ où $\zeta_R$ est la vorticité relative, $I_S = (1/S_z \! + \! 3)$, où
$S_z$ est le cisaillement vertical du vent entre le haut et le bas de la troposphère, $P_c$ sont les précipitations convectives sur océan seuillées sur $P_0$,
et enfin $k$ étant le coefficient de calibration de l'indice. Les constantes ajoutées au cisaillement et à la vorticité relative sont celles du SGP originel,
mentionnées dans la \cref{sec:intro_indices}, \cref{note:SGP} et constituent là le caractère empirique de l'indice. Comme son nom le suggère, la particularité
du CYGP est d'utiliser un potentiel convectif en lieu et place du potentiel thermique du SGP qui, dans \textcite{gray_tropical_1975} est défini avec la
stabilité atmosphérique (gradient vertical de température virtuelle équivalente), l'humidité relative moyenne entre \hPa{700} et \hPa{500} et l'énergie
thermique océanique accumulée sur les 60 premiers mètres sous la surface. Le CYGP de \citeauthor{royer_gcm_1998} a servi dans des applications en climat futur
par \textcite{mcdonald_tropical_2005,cattiaux_projected_2020,chauvin_response_2006,chauvin_future_2020}

L'indice le plus communément utilisé est sans aucun doute le GPI (\textit{Genesis Potential Index}) de \textcite{emanuel_tropical_2004} défini comme suit :
%
\begin{equation*}
    \tag*{GPI}
    \mathrm{GPI} = \underbrace{\vphantom{\left( \frac{H}{50} \right)^3}\lvert 10^5 \zeta \rvert^{3/2} \, (1 + 0.1 V_{\mathrm{shear}})^{-2}}_{\mathrm{Dynamique}}
    \underbrace{\left( \frac{H}{50} \right)^3 \left( \frac{V_{\mathrm{pot}}}{70} \right)^3}_{\mathrm{Thermique}}
\end{equation*}
%
Où $\zeta$ représente la vorticité absolue ($\zeta = \zeta_R + f$) exprimée en 10$^{-5}$~s$^{-1}$, $V_{\mathrm{shear}}$ l'amplitude du cisaillement vertical du
vent entre \hPa{850} et \hPa{200}, $H$ l'humidité relative à \hPa{700} et $V_{\mathrm{pot}}$ l'intensité potentielle (\textit{Potential Intensity}, PI)
\parencite{emanuel_airsea_1986,emanuel_sensitivity_1995,bister_dissipative_1998,bister_low_2002}. Cette dernière quantité remplace la SST dans l'indice, et
représente l'intensité maximale théorique à la surface d'un cyclone tropical, exprimée en m s$^{-1}$, lorsque ce dernier est assimilié à un cycle de Carnot
\parencite{emanuel_dependence_1987}. Sous cette hypothèse, le PI est proportionel à l'efficacité énergétique du cycle thermodynamique ---~exprimée comme la
différence de température aux deux extrémités du cycle~--- et de la source de chaleur. \textcite{bister_low_2002} ont alors montré que, sous certaines hypothèse
supplémentaires dont l'énumération dépasse le cadre de ce manuscrit, le PI pouvait être évalué comme suit :
%
\begin{equation*}
    \tag*{PI}
    V_{\mathrm{pot}}^2 = \frac{T_s}{T_0} \frac{C_k}{C_D} \left[ \mathrm{CAPE}^* - \mathrm{CAPE} \right] \rvert_{\mathrm{RMW}}
\end{equation*}
%
Où $T_s$ est la température de surface, $T_0$ la température de l'air rejeté au sommet du cyclone, $C_k$ est le coefficient d'échange d'enthalpie avec l'océan,
$C_D$ le coefficient de trainée, CAPE$^*$ l'énergie potentielle convective disponible (\textit{Convective Available Potential Energy}) de l'air saturé et élevé
jusqu'au niveau de l'air expulsé au sommet du cyclone, et CAPE étant l'énergie potentielle convective de la couche limite atmosphérique ---~CAPE$^*$ et CAPE
sont tous deux évalués au rayon de vent maximum (\textit{Radius of Maximum Wind}, RMW). En pratique, le calcul de l'intensité potentielle nécessite
l'utilisation d'un algorithme spécifique complexe et nécessite un temps de calcul conséquent.

\subsubsection{Modélisation statistique de l'activité cyclonique}

%--------------------------------------
\section{Exploration du potentiel prédictif d'un indice à l'échelle saisonnière}

\subsection{Choix du pas d'agrégation temporelle : Mensuel versus saisonnier}

\subsection{Activité observée versus modèle ARPEGE}

\subsection{Scores de performances sur la variabilité inter-annuelle}


%--------------------------------------
\section{Synthèse}

\end{document}
