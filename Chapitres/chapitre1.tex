% vim: spelllang=fr

\documentclass[../main.tex]{subfiles}
\graphicspath{{\subfix{../Figures/}}}
\begin{document}

\begin{itshape}
Ce premier chapitre introduit les cyclones tropicaux (TC), de la simple définition jusqu'à la formulation de la question scientifique présentée dans cette thèse, en introduisant tous les concepts intermédiaires nécessaires.
\end{itshape}

\minitoc
%----------------------------------------------------------------------------
\section{Introduction aux cyclones tropicaux}

\subsection{Qu'est-ce qu'un cyclone tropical}

Du grec \textit{κύκλος}, nom commum désignant un cercle, ou plus généralement toute chose cicurlaire ou ronde, le terme cyclone, dans un contexte météorologique, fait référence au type de circulation atmosphérique dans lequel l'air se trouve en rotation atour d'un centre de basse pression.

\subsection{Bassins d'activité et saisonnalité}

\subsection{Risques associés et enjeux}

%-------------------------------------------------------------------------------
\section{Ingrédients de la cyclogénèse}
  
\subsection{Conditions de formation}

\subsection{Modèles conceptuels de fonctionnement}

\subsubsection{CISK}

\subsubsection{WISHE}

%-------------------------------------------------------------------------------
\section{Cyclones tropicaux et changement climatique}

\subsection{Bases de données observationelles}

\subsection{Les cyclones dans les modèles de climat}

\subsection{Consensus actuel sur les projections futures}

\subsection{Détection objective v.s indices de cyclogénèse}

%-------------------------------------------------------------------------------
\section{Synthèse}

\end{document}
