% vim: spelllang=fr

\documentclass[../main.tex]{subfiles}
\begin{document}

Les cyclones tropicaux sont des larges systèmes dépressionnaires de plusieurs centaines de kilomètres de diamètre pouvant atteindre des vitesses de vents
soutenus jusqu'à \kmh{300}. Ils prennent naissance dans les tropiques et se répartissent dans six grands bassins océaniques majeurs. Ils font partie des
phénomènes naturels les plus destructeurs sur Terre. Bien que ces phénomènes prennent naissance et évoluent sur océan, certains peuvent atteindre les côtes et
provoquer des impacts dévastateurs, autant environnementaux que socio-économiques. Ils seraient directement responsables de la mort de plus de \num{400000}
morts sur la seule période \num{1980}~--~\num{2009}, dont les deux tiers sont attribués à seulement deux évènements \parencite{doocy_human_2013}. 

En raison de leur fort potentiel destructeur, ces phénomènes météorologiques sont étroitement surveillés. Les centres météorologiques des régions concernées
déploient par ailleurs de grands efforts pour prévoir à court terme leurs trajectoires et l'évolution de leur intensité, dans le but de pouvoir avertir la
population en cas de risque avéré. Néanmoins, la théorie sous-jacente à la formation des cyclones est mal connue, ce qui consistitue par conséquent une limite
importante à la prévisibilité de ces systèmes.

À défaut de comprendre le processus physique aboutissant à la formation d'un cyclone tropical ---~nommé processus de cyclogénèse~--- les conditions favorables à
leur émergence sont connues depuis de nombreuses années \parencite{gray_tropical_1975}. Une perturbation atmosphérique pré-éxistante pourra en effet donner lieu
à un cyclone tropical si les eaux océaniques sont suffisamment chaudes pour pouvoir y puiser une quantité suffisante d'énergie, d'une une atmosphère chargée en
humidité et d'une absence de cisaillement vertical du vent.

Les émissions de gaz à effet de serre aboutissent à un forçage radiatif qui déséquilibre le système climatique \parencite{charney_carbon_1979}, aboutissant
entre autres à une augmentation de la température moyenne planétaire. Les changements qui en émergent sont le résultat de rétro-actions particulièrement
complexes et dont les conséquences exactes à venir sont largement entâchées d'incertitudes encore à ce jour \parencite{seneviratne_weather_2021}. Le
réchauffement climatique amène donc à des changements dans les conditions favorables à l'émergence de cyclones tropicaux et a par conséquent un impact sur
l'activité cyclonique.

Néanmoins, l'incertitude dans les conséquences du forçage radiatif d'origine anthropique sur le système climatique, couplée à l'incertitude sur le processus de
cyclogénèse, rend difficile l'estimation de l'effet du réchauffement climatique sur l'activité cyclonique. À cela s'ajoutent une certaine hétérogénéité dans
le temps et l'espace des séries d'observations historiques ---~compliquant la recherche d'éventuelles tendances en cours dans la fréquence d'occurrence des
cyclones~--- ainsi qu'une résolution spatiale des modèles climatiques encore trop grossière pour simuler fidèlement ces systèmes.

Il existe toutefois deux options pour chercher à évaluer l'activité cyclonique tropicale future. La première, que l'on peut qualifier d'approche directe,
consiste à appliquer un algorithme de détection et de suivi des cyclones tropicaux dans des simulations climatiques à haute résolution, de préférence à la plus
haute résolution possible. La seconde option consiste à exploitier les simulations climatiques à basse résolution exsistantes en cherchant des liens
statistiques entre 

\end{document}
