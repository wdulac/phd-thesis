% vim: spelllang=fr

\documentclass[../main.tex]{subfiles}
\begin{document}

Les cyclones tropicaux sont des larges systèmes dépressionnaires de plusieurs centaines de kilomètres de diamètre pouvant atteindre des vitesses de vents
soutenus jusqu'à \kmh{300}. Ils prennent naissance dans les tropiques et se répartissent dans six grands bassins océaniques majeurs. Ils font partie des
phénomènes naturels les plus destructeurs sur Terre. Bien qu'ils évoluent sur océan, certains peuvent atteindre les côtes et avoir des impacts dévastateurs,
autant environnementaux que socio-économiques. Ils seraient directement responsables de la mort de plus de \num{400000} personnes sur la seule période
\num{1980}~--~\num{2009}, dont les deux tiers sont attribués à seulement deux évènements \parencite{doocy_human_2013}. En raison de leur fort potentiel
destructeur, ces phénomènes météorologiques sont étroitement surveillés. Les centres météorologiques des régions concernées déploient par ailleurs de grands
efforts pour prévoir à court terme leurs trajectoires et l'évolution de leur intensité, dans le but de pouvoir avertir la population en cas de risque avéré.
Néanmoins, la théorie sous-jacente à la formation des cyclones est mal connue, ce qui constitue par conséquent une limite importante à la prévisibilité de ces
systèmes. À défaut de comprendre le processus physique aboutissant à la formation d'un cyclone tropical ---~nommé processus de cyclogénèse~--- les conditions
favorables à leur émergence sont connues depuis de nombreuses années \parencite{gray_tropical_1975}. Une perturbation atmosphérique pré-éxistante pourra en
effet donner lieu à un cyclone tropical si les eaux océaniques sont suffisamment chaudes pour pouvoir y puiser une quantité suffisante d'énergie, d'une
atmosphère chargée en humidité et d'une absence de cisaillement vertical du vent.

%Les changements qui en émergent sont le résultat de rétro-actions particulièrement
%complexes et dont les conséquences exactes à venir sont largement entâchées d'incertitudes encore à ce jour \parencite{seneviratne_weather_2021}.

Les émissions de gaz à effet de serre induisent un forçage radiatif qui déséquilibre le système climatique \parencite{charney_carbon_1979}, aboutissant entre
autres à une augmentation de la température moyenne planétaire. Il est en outre certain que le réchauffement climatique amène à des changements dans la
répartition des conditions environnementales jugées favorables à l'émergence de cyclones tropicaux et, donc que celui-ci a un impact sur l'activité cyclonique
\parencite{seneviratne_weather_2021}. Néanmoins, l'incertitude dans les effets exacts du forçage radiatif d'origine anthropique sur le système climatique,
couplée à l'incertitude sur le processus de cyclogénèse, rend difficile l'estimation de l'effet du réchauffement climatique sur l'activité cyclonique. À cela
s'ajoutent une certaine hétérogénéité dans le temps et l'espace des séries d'observations historiques ---~compliquant la recherche d'éventuelles tendances en
cours dans la fréquence d'occurrence des cyclones~--- ainsi qu'une résolution spatiale de la plupart des modèles climatiques encore trop grossière pour simuler fidèlement ces
systèmes. Il existe toutefois deux options pour chercher à évaluer l'activité cyclonique tropicale future. La première, que l'on peut qualifier d'approche
directe, consiste à appliquer un algorithme de détection et de suivi des cyclones tropicaux dans des simulations climatiques à haute résolution. La seconde
option consiste à exploiter les simulations climatiques à basse résolution en cherchant des liens statistiques entre l'activité observée et l'environnement de
grande échelle. Il s'agit alors d'indices de cyclogénèse. Cette seconde approche peut être qualifiée d'indirecte, en ce sens que l'activité cyclonique est
simplement inférée des champs de grande échelle issues des simulations climatiques.

Ces deux approches tendent à s'accorder sur certains points, comme par exemple sur une probable augmentation de l'intensité maximale des cyclones
\parencite{sobel_human_2016,bhatia_projected_2018}, mais sont généralement en désaccord sur la question de la fréquence d'occurrence des cyclones tropicaux en
climat futur. En effet, une diminution de l'activité cyclonique constitue un signal robuste de réponse au réchauffement climatique à travers les modèles pour
l'approche directe \parencite{christensen_climate_2013,knutson_tropical_2020}. De l'autre côté, les indices de cyclogénèse tendent au contraire à indiquer une
augmentation de la fréquence d'occurrence des cyclones tropicaux \parencite{emanuel_downscaling_2013,camargo_testing_2014}. Outre ce désaccord, chacune des deux
approches possède son lot de points positifs et de points négatifs.

L'utilisation d'un algorithme de détection de cyclones tropicaux présente certes l'avantage non-négligeable de constituer une mesure directe, mais la conception
même de l'algorithme, ses divers paramètres qui nécessitent souvent d'être ajustés à chaque modèle, voire à chaque simulation, et autres spécificités
constituent autant de sources d'incertitudes qui viennent s'ajouter aux biais propres aux modèles. Ces deux sources d'incertitudes deviennent alors
indiscernables l'une de l'autre. En effet, les algorithmes de détection consistent en l'application objective d'un schéma conceptuel de fonctionnement d'un
cyclone tropical. Comment peut-on alors savoir si un défaut, ou un surplus de cyclones dans une simulation provient du modèle utilisé ou du traqueur ?
Inversement, comment peut-on savoir si les systèmes détectés par le traqueur correspondent toujours bien à des cyclones tropicaux ? Ces deux questions peuvent
finalement être regroupées en une seule : Comment peut-on évaluer les performances d'un algorithme de détection de cyclones tropicaux ?

Les indices de cyclogénèse, quant à eux, présupposent que les relations statistiques unissant l'environnement de grande échelle à l'activité observée sont
stationnaires dans le temps ---~une hypothèse dont la validité n'est en aucune façon triviale \parencite{nolan_increased_2008,murakami_changes_2013}. De plus,
si les indices de cyclogénèse excellent dans leur capacité à simuler une climatologie spatiale et saisonnière réaliste, la variabilité interannuelle simulée par
ces derniers est souvent incohérente avec les observations. Il est possible que la raison à cette mauvaise variabilité interannuelle soit due à une mauvaise
représentation du processus physique sous-jacent. Or, un indice de cyclogénèse qui fournirait une bonne variabilité interannuelle en raison d'une meilleure
représentation du processus aurait plus de chances de projeter la bonne réponse en changement climatique. Comment alors améliorer la représentation de la
variabilité interannuelle simulée par les indices de cyclogénèse ?

Pour apporter quelques éléments de réponse à ces questions, cette thèse s'organise de la façon suivante : Le \cref{chap:chapitre_1} est consacré à un grand tour
d'horizon sur les cyclones tropicaux. Dans ce chapitre, les cyclones tropicaux sont introduits, leur climatologie détaillée et les risques et enjeux associés
sont précisés. Ce chapitre introduit également les concepts nécessaires à la bonne compréhension de la suite du manuscrit. En particulier, il y est question des
bases de données observationnelles ainsi que des tendances connues à ce jour. Le fonctionnement général d'un modèle de climat est décrit, avec une attention
plus particulière sur la manière dont les modèles sont utilisés pour étudier les cyclones tropicaux. Les deux approches utilisées dans la littérature pour
évaluer l'activité cyclonique future et les problématiques associées sont introduits plus en détails, avant de se terminer sur le consensus scientifique actuel
sur l'évolution future de l'activité cyclonique tropicale.

Le \cref{chap:chapitre_2} est entièrement consacré à l'approche directe par détection objective de cyclones dans les modèles, et plus précisément à
l'utilisation de l'algorithme de détection du CNRM. Dans une première partie du chapitre, le traqueur est appliqué à une réanalyse atmosphérique ---~c'est à
dire à une simulation spécialement conçue pour représenter le plus fidèlement possible l'état passé de l'atmosphère, d'après les observations historiques
disponibles~--- puis évalué à l'aide de la base de données des cyclones tropicaux historiques. Ce cadre de travail permet en effet de caractériser au mieux les
propriétés du traqueur. Une seconde partie du chapitre porte sur des considérations supplémentaires, notamment sur les méthodes disponibles de filtrage des
systèmes indésirables détectées par les traqueurs, ainsi que sur la mise au point de métriques d'évaluation alternatives de la qualité des trajectoires
détectées par un algorithme de suivi.

Le \cref{chap:chapitre_3} introduit en détail l’approche indirecte consistant en l’utilisation d’indices de cyclogénèse pour évaluer l’activité cyclonique dans
les modèles. Les indices les plus couramment utilisés dans la littérature sont présentés et évalués dans ERA5, et un bref historique des différents
développements est présenté. Une attention toute particulière est prêtée à un indice dont la méthodologie consiste en l'application d'une régression statistique
entre l'environnement de grande échelle et l'activité observée. Cet indice présente en effet l'intérêt d'être facilement reproductible. Cette méthodologie est
alors employée pour étudier l'effet de divers facteurs sur les coefficients de pondération de l'indice, ainsi que pour étudier la manière dont ces coefficients
affectent la représentation de la variabilité interannuelle de l'indice.

Enfin, dans le \cref{chap:chapitre_4}, nous nous intéressons à l’ajout de nouveaux prédicteurs dans les indices de cyclogénèse construits selon cette
méthodologie. L’apport de l’ajout explicite du mode de variabilité El Niño dans la régression est étudié dans un premier temps. Ensuite, nous nous intéressons
au remplacement du prédicteur d'humidité relative par une mesure intégrée sur la colonne du déficit de saturation d’humidité. Cet indice est ensuite appliqué à
deux simulations en climat plus chaud réalisées avec le modèle ARPEGE-Climat dans une configuration à résolution variable permettant d'atteindre localement de
très hautes résolutions.

\end{document} 
