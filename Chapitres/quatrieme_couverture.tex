\documentclass[../main.tex]{subfiles}
\begin{document}

\cleartoevenpage[\thispagestyle{empty}]
\thispagestyle{empty}

\newgeometry{top=1cm, bottom=2cm, left=2.5cm, right=2.5cm}
%\vspace*{\fill}
\begin{center}
\Large\textbf{Methodes d'évaluation de l'activité cyclonique tropicale en changement climatique\\}
\end{center}
\vspace{1cm}

	\begin{minipage}[t]{0.495\textwidth}

		\large{\textbf{Auteur :}} \\
		\large{\textbf{Directeur de thèse :}} \\
  		\large{\textbf{Co-encadrant de thèse :}}

	\end{minipage}
	\hfill{}
	\begin{minipage}[t]{0.495\textwidth}

		\large{\textbf{William DULAC}} \\
		\large{\textbf{Julien CATTIAUX}} \\
  		\large{\textbf{Fabrice CHAUVIN}}\\

	\end{minipage}

\vspace{1cm}
\noindent\rule[2pt]{\textwidth}{0.5pt}
\noindent{\large\textbf{Résumé court}} --- Deux méthodes existent pour évaluer l’activité cyclonique tropicale en changement climatique : L’utilisation
d’algorithmes de détection (traqueurs) de cyclones appliqués aux modèles de climat, ainsi que l’utilisation d’indices de cyclogénèse ; des relations
statistiques liant l’activité cyclonique observée à des variables atmosphériques de grande échelle. Ces deux méthodes tendent à fournir des projections opposées
dans les simulations climatiques. Les travaux réalisés dans cette thèse explorent chacune de ces deux méthodes d’évaluation dans le but d’apporter des
améliorations à chacune d’elles. Le traqueur du CNRM est caractérisé en détails à travers une application à la réanalyse ERA5, et de nouvelles métriques
d’évaluation des performances de détection sont proposées. De nouveaux indices de cyclogénèse sont construits et évalués, avec comme objectif principal
d’améliorer la représentation de leur variabilité interannuelle.\\
\noindent{\large\textbf{Mots clés :}}
    Cyclones tropicaux, algorithme de détection, indices de cyclogénèse, changement climatique 
\\
\noindent\rule[2pt]{\textwidth}{0.5pt}

\vspace{5mm}
\selectlanguage{english}
\noindent\rule[2pt]{\textwidth}{0.5pt}
\noindent{\large\textbf{Short abstract}} --- Two methods exist for assessing tropical cyclone activity under climate change: the use of cyclone detection
algorithms (trackers) applied to climate models, and the use of cyclogenesis indices; statistical relationships linking observed cyclone activity to large-scale
atmospheric variables. These two methods tend to provide opposite projections in climate simulations. The work carried out in this thesis explores each of
these two assessment methods, with the aim of making improvements to each. The CNRM tracker is characterized in detail through an application to ERA5
reanalysis, and new metrics for evaluating detection performance are presented. New cyclogenesis indices are constructed and evaluated, with the main aim of
improving the representation of their interannual variability.\\
\noindent{\large\textbf{Keywords:}}
    Tropical cyclones, tracking algorithm, cyclogenesis indices, climate change
\\
\noindent\rule[2pt]{\textwidth}{0.5pt}

\vspace{1cm}
\noindent{\large\textbf{École doctorale :}}
SDU2E -- Sciences de l'Univers, de l'Environnement et de l'Espace

\noindent{\large\textbf{Spécialité :}}
Océan, Atmosphère, Climat
\\

\vspace{6mm}
\begin{center}
  Centre National de Recherches Météorologiques, 42 avenue Gaspard Coriolis\\
  31100 Toulouse
\end{center}

\end{document}
